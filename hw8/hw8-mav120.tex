\documentclass[12pt]{article}
 
\usepackage[margin=1in]{geometry} 
\usepackage{amsmath,amsthm,amssymb}
\usepackage{graphicx}
\usepackage{enumitem} 

\begin{document}

\title{CS 1555 HW 8}
\author{Mathew Varughese - mav120} 
\date{Wednesday, Apr 1}

\maketitle

\begin{enumerate}

\item % 1

\begin{enumerate}

\item % 1a

Primary Keys are ItemId and PurchaceId. These are 
the ones that do not appear on the left hand side


I will find these using the universal relational approach.


Using ItemId $\rightarrow$ ItemDescription, 

\hspace{10mm} R1(\underline{ItemId}, \underline{PurchaceId}, PurchaceDate, VendorCode, VendorName, VendorAddress, OrderQuantity, ItemPrice, StockQuantity)

\hspace{10mm} R2(\underline{ItemId}, ItemDescription)

\vskip 12pt

Using ItemId $\rightarrow$ ItemPrice

\hspace{10mm} R11(\underline{ItemId}, \underline{PurchaceId}, PurchaceDate, VendorCode, VendorName, VendorAddress, OrderQuantity, StockQuantity)

\hspace{10mm} R12(\underline{ItemId}, ItemPrice)

\vskip 12pt

Using ItemId $\rightarrow$ StockQuantity

\hspace{10mm} R111(\underline{ItemId}, \underline{PurchaceId}, PurchaceDate, VendorCode, VendorName, VendorAddress, OrderQuantity)

\hspace{10mm} R112(\underline{ItemId}, StockQuantity)

\vskip 12pt

Using PurchaceId $\rightarrow$ PurchaceDate

\hspace{10mm} R1111(\underline{ItemId}, \underline{PurchaceId}, VendorCode, VendorName, VendorAddress, OrderQuantity)

\hspace{10mm} R1112(\underline{PurchaceId}, PurchaceDate)

\vskip 12pt

Using PurchaceId $\rightarrow$ VendorCode

\hspace{10mm} R11111(\underline{ItemId}, \underline{PurchaceId}, VendorName, VendorAddress, OrderQuantity)

\hspace{10mm} R11112(\underline{PurchaceId}, VendorCode)

\vskip 12pt

Using VendorCode $\rightarrow$ VendorName

\hspace{10mm} R111111(\underline{ItemId}, \underline{PurchaceId}, VendorAddress, OrderQuantity)

\hspace{10mm} R111112(\underline{VendorCode}, VendorName)

\vskip 12pt

Using VendorCode $\rightarrow$ VendorAddress

\hspace{10mm} R1111111(\underline{ItemId}, \underline{PurchaceId}, OrderQuantity)

\hspace{10mm} R1111112(\underline{VendorCode}, VendorAddress)

\vskip 12pt

Using ItemId $\rightarrow$ OrderQuantity

\hspace{10mm} R11111111(\underline{ItemId}, \underline{PurchaceId})

\hspace{10mm} R11111112(\underline{ItemId}, OrderQuantity)

\vskip 12pt

So, our relations are 

R1(\underline{ItemId}, \underline{PurchaceID})

R2(\underline{VendorCode}, VendorName, VendorAddress)

R3(\underline{PurchaceID}, PurchaceDate, VendorCode)

R4(\underline{ItemId}, ItemDescription, ItemPrice, StockQuantity)

R5(\underline{ItemId}, OrderQuantity)

\item % 1b
\vskip 12pt
This composition is good, because each table depends on the primary key and nothing but that key. The construction is indeed lossless, as R1 has access to all attributes.
\begin{table}[h]
\resizebox{\textwidth}{!}{%
\begin{tabular}{|l|l|l|l|l|l|l|l|l|l|l|l|}
\hline
  & ItemId & ItemDesc & ItemPrice & StockQuantity & PurchaceId & PurchaceDate & VendorCode & VendorName & VendorAddress & OrderQuantity &  \\ \hline
R1 & A1 & A2 & A3 & A4 & A5 & A6 & A7 & A8 & A9 & A10 \\ \hline
R2 & U  & U  & U  & U  & U  & U  & A7 & A8 & A9 & U   \\ \hline
R3 & U  & U  & U  & U  & A5 & A6 & A7 & A8 & A9 & U   \\ \hline
R4 & A1 & A2 & A3 & A4 & U  & U  & U  & U  & U  & U   \\ \hline
R5 & A1 & A2 & A3 & A4 & U  & U  & U  & U  & U  & A10   \\ \hline
\end{tabular}}
\caption{This is the table to check for lossless join created by applying each functional decomp}
\label{tab:oneb}
\end{table}

\end{enumerate}


\item % 2

\begin{enumerate}

\item % 2a

A $\rightarrow$ B

B $\rightarrow$ CD

A $\rightarrow$ D

B $\rightarrow$ C

AB $\rightarrow$ CD

A $\rightarrow$ C

E $\rightarrow$ F

\hrulefill

Transform all FDs to canonical form

A $\rightarrow$ B

B $\rightarrow$ C

B $\rightarrow$ D

A $\rightarrow$ D

B $\rightarrow$ C

AB $\rightarrow$ C

AB $\rightarrow$ D

A $\rightarrow$ C

E $\rightarrow$ F

\hrulefill

Remove Redudancies

A $\rightarrow$ B

B $\rightarrow$ C

B $\rightarrow$ D

A $\rightarrow$ D

B $\rightarrow$ C

AB $\rightarrow$ C

AB $\rightarrow$ D

A $\rightarrow$ C

E $\rightarrow$ F

\hrulefill

Drop Extraneous attributes

A $\rightarrow$ B

B $\rightarrow$ C

B $\rightarrow$ D

E $\rightarrow$ F

\hrulefill

Finding keys. A, E does not appear in right hand keys, so it must appear in all keys.

AE+ : AE $\rightarrow$ AEB $\rightarrow$ AEBC  $\rightarrow$ AEBCDF

So key is AE. 

\hrulefill

Group FDs with same determinant

A $\rightarrow$ B

B $\rightarrow$ CD

E $\rightarrow$ F

\hrulefill

Construct relation 

R1 (\underline{A}, B)

R2 (\underline{B}, C, D)

R3 (\underline{E}, F)


\hrulefill

If none of the relations contain the key for the original relation add a relation with the key.

R1 (\underline{A}, B)

R2 (\underline{B}, C, D)

R3 (\underline{E}, F)

R4 (\underline{A}, \underline{E})


\item % 2b 

This is lossless because it covers all attributes

\begin{table}[h]
	\centering
	\begin{tabular}{|l|l|l|l|l|l|l|}
	\hline
	  & A  & B  & C  & D  & E  & F  \\ \hline
	R1 & A1 & A2 & A3 & A4 & U  & U  \\ \hline
	R2 & U  & A2 & A3 & A4 & U  & U  \\ \hline
	R3 & U  & U  & U  & U  & A5 & A6 \\ \hline
	R4 & A1 & A2 & A3 & A4 & A5 & A6 \\ \hline
	\end{tabular}
	\end{table}
	
\end{enumerate}

\end{enumerate}

\end{document}