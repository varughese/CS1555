\documentclass[12pt]{article}
 
\usepackage[margin=1in]{geometry} 
\usepackage{amsmath,amsthm,amssymb}
\usepackage{enumitem} 

\begin{document}

\title{CS 1555 HW 3}
\author{Mathew Varughese - mav120} 
\date{Wednesday, Feb 5}

\maketitle

I assume that Payments.cell\_pn is a foreign key to the primary key Customers.cell\_pn .

\begin{enumerate}

\item

\begin{enumerate}

\item
Arity = 1
Cardinality = 10

\item
Arity = 1
Min Cardinality = 1
Max Cardinality = 10

\item
Arity = 14
Min Cardinality = 10
Max Cardinality = 17

It depends on what the foreign key of cell\textunderscore pn comes from. 
I assume the
cell\textunderscore pn attribue in PAYMENTS is a foreign key to CUSTOMERS.

\item
Arity = 15 
Min Cardinality = 0
Max Cardinality = 10

Because cell\_pn and paid\_on are a unique key, you cannot
have multiple cell\_pn with the same date of '2019-09-01', so at the maximum, you
can have 10 rows on PAYMENTS that have different cell phone numbers, that join with
cell\_pn on CUSTOMERS.

\end{enumerate}


\item

\begin{enumerate}

\item
$ \Pi_{total\_minutes, total\_SMS}(\sigma_{city = 'philadelphia}(Customers) * Statements)$

By selectivity, we can reduce the rows in the Customers, 
which will make the natural join with Statements more efficient.
The time of the projection does not matter for time complexity.

\end{enumerate}

\item

\begin{enumerate}
	
\item
$ \Pi_{fname, lname}(\sigma_{city='Pittsburgh'}(Customers)) $

\item
Retrieve the phone numbers of customers who made calls to people in Pittsburgh.

\noindent $ \Pi_{cell\_pn} (Customers \bowtie_{cell\_pn = from\_pn} ((\sigma_{city='Pittsburgh'} Directory) 
\bowtie_{Directory.pn = Records.to\_pn} RECORDS \ )) $

\item
List the SSNs of all customers that have ever paid more than 100 in a single payment,
and have ever had an amount due more than 50.

\noindent $ \Pi_{SSN}(Customers * (\sigma_{amount\_paid > 100}(Payments) 
* \sigma_{amount\_due > 50}(Statements)) ) $

\end{enumerate}

\item

\begin{enumerate}

\item 
List only once every pair of cell phone numbers which use the same number of SMS
in July 2019. (I am assuming that $ \leq $ '2019-07-31' can find all records before the last day
in July)

$ JULY \leftarrow \sigma_{start\_date \ \geq \ '2019-07-01' \ \land \ end\_date \ \leq \ '2019-07-31'} Statements $

$ JSMS \leftarrow \Pi_{cell\_pn, total\_SMS} (JULY) $

$ JSMS2 \leftarrow \rho_{cell\_pn2, total\_SMS} (JSMS) $

$ BOTH \leftarrow JSMS * JSMS2 $

$ REMOVEDUPS \leftarrow \sigma_{cell\_pn > cell\_pn2} BOTH $

$ RSLT \leftarrow \Pi_{cell\_pn, cell\_pn2} REMOVEDUPS $

\item
Find the SSNs of all customers who received calls from people in Pennsylvania, where
they have at least one call duration more than 20.

Assuming this means, a customer is shown if they have received
a call from Pennsylvania and at least one of those calls was 
longer than 20. 


$ PA \leftarrow (\sigma_{state='PA'}(Directory)) $

$ DUR \leftarrow (\sigma_{duration > 20}(Records)) $

$ PARECORDS \leftarrow PA \bowtie_{Directory.pn = Records.from\_pn} DUR $ 

$ CUS \leftarrow PARECORDS \bowtie_{Records.to\_pn = Customers.cell\_pn} CUSTOMERS $

$ RSLT \leftarrow \Pi_{SSN}(CUS) $

\item
List SSNs for all customers that live in Pittsburgh city and received calls
from New York state but never made calls to New York State

$ PGH \leftarrow \sigma_{city='Pittsburgh'}(Customers) $

$ TO\_PGH \leftarrow PGH \bowtie_{Customers.cell\_pn = Records.to\_pn} Directory $

$ FROM\_PGH \leftarrow PGH \bowtie_{Customers.cell\_pn = Records.from\_pn} Directory $

$ NY \leftarrow \sigma_{state='NY'} Directory $

$ NOT\_NY \leftarrow \sigma_{state \neq 'NY'} Directory $

$ FROMNY \leftarrow TO\_PGH \bowtie_{Records.from\_pn = Directory.pn} NY $

$ TO\_NOT\_NY \leftarrow FROM\_PGH \bowtie_{Records.to\_pn = Directory.pn} NOT\_NY $

$ SSN\_FROMNY \leftarrow \Pi_{SSN} FROMNY $

$ SSN\_TO\_NOT\_NY \leftarrow \Pi_{SSN} TO\_NOT\_NY $

$ RSLT \leftarrow SSN\_FROMNY \cap SSN\_TO\_NOT\_NY $

\end{enumerate}


\end{enumerate}

\end{document}