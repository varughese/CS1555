\documentclass[12pt]{article}
 
\usepackage[margin=1in]{geometry} 
\usepackage{amsmath,amsthm,amssymb}
\usepackage{enumitem} 

\begin{document}

\title{CS 1555 HW 4}
\author{Mathew Varughese - mav120} 
\date{Wednesday, Feb 12}

\maketitle

\begin{enumerate}

\item % 1

\begin{enumerate}

\item % a

Arity = 1, Min = 1, Max = 1
% $ \,_2F_1 $

$ J \leftarrow \sigma_{start\_timestamp \geq '2019-08-01 \ 00:00:00' \land start\_timestamp < '2019-09-01 \ 00:00:00' } (RECORDS) $

$ PA \leftarrow J \bowtie_{from\_pn = cell\_pn} (\sigma_{state='PA'} DIRECTORY) $

$ RSLT \leftarrow F_{max(Duration)} PA $

\item % b
Arity = 2 (zip, avg(payment\_count)), Min = 0 (if there are no statements in Nov 2019), Max = 50 (50 different zip codes)

$ S \leftarrow \sigma_{start\_date \geq '2019-11-01 \ 00:00:00' \land end\_date < '2019-12-01 \ 00:00:00' } (Statements) $

$ R \leftarrow \Pi_{cell\_pn} S$

$ COUNT(zip, payment\_count) \leftarrow _{zip}F_{count(cell\_pn)} R $

$ RSLT \leftarrow _{zip}F_{avg(payment\_count)} $

\item % c
Arity = 2, Min = 0, Max = 25 (every customer could have 2 cell\_pns, so that 
would mean 50/2, or 25 maximum cell\_pns)

$ S(SSN, cell\_pn\_count) \leftarrow _{SSN}F_{count(cell\_pn)} $

$ R \leftarrow \sigma_{cell\_pn\_count > 1} S $

$ RSLT \leftarrow \Pi_{fname, lname} (R * Customers) $

\item % d
Assumption: If the there is only one Customer with a last name, we consider
no one else in their family to be a customer of P\_Mobile

Assumption: The count function does not count duplicates. For example, if there 
are two SSNs that are the same, $COUNT(SSN)$ would return 1. 

Arity = 1, Min = 0, Max = 50

$ C(lname, count\_ssn) \leftarrow _{lname} F_{count(SSN)} Customers $

$ RSLT \leftarrow \Pi_{lname} \ (\sigma_{count\_ssn = 1} C ) $

\item % e
Arity = 1, Min = 0, Max = 1

This question is very vague, so I will assume that it is asking to find
the sum of the charges for this customer.

$ C \leftarrow \sigma_{cell\_pn = 4129876543 \ \land \ start\_date \ \geq \  '1-01-2019'} (Statements) $

$ RSLT \leftarrow F_{sum(amount\_due)} C $


\end{enumerate}

\item % 2

See next page


\begin{table}[]
	\caption{R}
	\centering
	\begin{tabular}{|l|l|l|l|}
	\hline
	\textbf{A} & \textbf{B} & \textbf{C} & \textbf{D} \\ \hline
	0          & 0          & 0          & 1          \\ \hline
	0          & 0          & 0          & 2          \\ \hline
	0          & 0          & 0          & 3          \\ \hline
	0          & 0          & 0          & 4          \\ \hline
	0          & 0          & 0          & 5          \\ \hline
	0          & 0          & 0          & 8          \\ \hline
	0          & 0          & 0          & 9          \\ \hline
	0          & 0          & 0          & 10         \\ \hline
	0          & 0          & 0          & 11         \\ \hline
	0          & 0          & 0          & 12         \\ \hline
	0          & 0          & 0          & 13         \\ \hline
	0          & 0          & 0          & 14         \\ \hline
	0          & 0          & 0          & 15         \\ \hline
	\end{tabular}
	\end{table}



\begin{table}[]
	\caption{S}
	\centering
	\begin{tabular}{|l|l|l|}
	\hline
	\textbf{D} & \textbf{E} & \textbf{F} \\ \hline
	1          & 0          & 0          \\ \hline
	2          & 0          & 0          \\ \hline
	3          & 0          & 0          \\ \hline
	4          & 0          & 0          \\ \hline
	5          & 0          & 0          \\ \hline
	6          & 0          & 0          \\ \hline
	7          & 0          & 0          \\ \hline
	\end{tabular}
	\end{table}

\begin{table}[]
	\caption{R right-outer-join(R.D = S.D) S}
	\centering
	\begin{tabular}{|l|l|l|l|l|l|l|}
	\hline
	\textbf{R.A} & \textbf{R.B} & \textbf{R.C} & \textbf{R.D} & \textbf{S.D} & \textbf{S.E} & \textbf{S.F} \\ \hline
	0            & 0            & 0            & 1            & 1            & 0            & 0            \\ \hline
	0            & 0            & 0            & 2            & 2            & 0            & 0            \\ \hline
	0            & 0            & 0            & 3            & 3            & 0            & 0            \\ \hline
	0            & 0            & 0            & 4            & 4            & 0            & 0            \\ \hline
	0            & 0            & 0            & 5            & 5            & 0            & 0            \\ \hline
	0            & 0            & 0            & 8            & null         & null         & null         \\ \hline
	0            & 0            & 0            & 9            & null         & null         & null         \\ \hline
	0            & 0            & 0            & 10           & null         & null         & null         \\ \hline
	0            & 0            & 0            & 11           & null         & null         & null         \\ \hline
	0            & 0            & 0            & 12           & null         & null         & null         \\ \hline
	0            & 0            & 0            & 13           & null         & null         & null         \\ \hline
	0            & 0            & 0            & 14           & null         & null         & null         \\ \hline
	0            & 0            & 0            & 15           & null         & null         & null         \\ \hline
	null         & null         & null         & null         & 6            & 0            & 0            \\ \hline
	null         & null         & null         & null         & 7            & 0            & 0            \\ \hline
	\end{tabular}
	\end{table}



\end{enumerate}

\end{document}