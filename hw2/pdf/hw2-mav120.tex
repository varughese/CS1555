\documentclass[12pt]{article}
 
\usepackage[margin=1in]{geometry} 
\usepackage{amsmath,amsthm,amssymb}
\usepackage{enumitem} 

\newcommand{\problem}[1]{
	\vskip 1em
	{\noindent \large \textbf{#1}}
}

\begin{document}

\title{CS 1555 HW 2}
\author{Mathew Varughese - mav120} 
\date{Wednesday, Jan 29}

\maketitle




\problem{1.}

Assuming cell phone is not null (There are no customers that would not have a cell phone number)


\begin{verbatim}
CUSTOMERS
PK(SSN) ; UQ(cell_pn)
FK(cell_pn) -> DIRECTORY(pn)
FK(home_pn) -> DIRECTORY(pn)
\end{verbatim}

Assuming that you can SMS and Call at the same time stamp, which is why TYPE is 
needed in the PK for RECORDS.
I assume that whenever a call is made, this phone number will be added to the directory before a new record in the RECORDS table.

\begin{verbatim}
RECORDS
PK(from_pn, to_pn, start_timestamp, type)
FK(from_pn) -> DIRECTORY(pn)
FK(to_pn) -> DIRECTORY(pn)
\end{verbatim}


\begin{verbatim}
STATEMENTS
PK(cell_pn, start_date, end_date)
FK(cell_pn) -> DIRECTORY(pn)
\end{verbatim}

\begin{verbatim}
PAYMENTS
PK(cell_pn, paid_on, amount_paid)
\end{verbatim}

Assuming that phone number is unique in the DIRECTORY table. 
For phone numbers that are home phone numbers, we only list the first name
and last name of the owner of that home phone number. For example, in the sample data 
4123121231 is the home phone number for Kate and Bill Stevenson, but in the directory this is
only listed once. 

\vskip 1em

I assume that it is okay for the address in DIRECTORY table to be different than that in the CUSTOMERS table. This is because I assume this DIRECTORY has home addresses, whereas CUSTOMERS has billing addresses. Most of the time they are the same, but they can be different.  

\begin{verbatim}
DIRECTORY
PK(pn)
\end{verbatim}

\end{document}